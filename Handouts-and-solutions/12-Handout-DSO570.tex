



    
% !TeX spellcheck = de_DE
% !TeX encoding = UTF-8
\documentclass{scrreprt}

    
    
    \usepackage[T1]{fontenc}
    % Nicer default font (+ math font) than Computer Modern for most use cases
    \usepackage{mathpazo}

    % Basic figure setup, for now with no caption control since it's done
    % automatically by Pandoc (which extracts ![](path) syntax from Markdown).
    \usepackage{graphicx}
    % We will generate all images so they have a width \maxwidth. This means
    % that they will get their normal width if they fit onto the page, but
    % are scaled down if they would overflow the margins.
    \makeatletter
    \def\maxwidth{\ifdim\Gin@nat@width>\linewidth\linewidth
    \else\Gin@nat@width\fi}
    \makeatother
    \let\Oldincludegraphics\includegraphics
    % Set max figure width to be 80% of text width, for now hardcoded.
    \renewcommand{\includegraphics}[1]{\Oldincludegraphics[width=.8\maxwidth]{#1}}
    % Ensure that by default, figures have no caption (until we provide a
    % proper Figure object with a Caption API and a way to capture that
    % in the conversion process - todo).
    \usepackage{caption}
    \DeclareCaptionLabelFormat{nolabel}{}
    \captionsetup{labelformat=nolabel}

    \usepackage{adjustbox} % Used to constrain images to a maximum size 
    \usepackage{xcolor} % Allow colors to be defined
    \usepackage{enumerate} % Needed for markdown enumerations to work
    \usepackage{geometry} % Used to adjust the document margins
    \usepackage{amsmath} % Equations
    \usepackage{amssymb} % Equations
    \usepackage{textcomp} % defines textquotesingle
    % Hack from http://tex.stackexchange.com/a/47451/13684:
    \AtBeginDocument{%
        \def\PYZsq{\textquotesingle}% Upright quotes in Pygmentized code
    }
    \usepackage{upquote} % Upright quotes for verbatim code
    \usepackage{eurosym} % defines \euro
    \usepackage[mathletters]{ucs} % Extended unicode (utf-8) support
    \usepackage[utf8x]{inputenc} % Allow utf-8 characters in the tex document
    \usepackage{fancyvrb} % verbatim replacement that allows latex
    \usepackage{grffile} % extends the file name processing of package graphics 
                         % to support a larger range 
    % The hyperref package gives us a pdf with properly built
    % internal navigation ('pdf bookmarks' for the table of contents,
    % internal cross-reference links, web links for URLs, etc.)
    \usepackage{hyperref}
    \usepackage{longtable} % longtable support required by pandoc >1.10
    \usepackage{booktabs}  % table support for pandoc > 1.12.2
    \usepackage[inline]{enumitem} % IRkernel/repr support (it uses the enumerate* environment)
    \usepackage[normalem]{ulem} % ulem is needed to support strikethroughs (\sout)
                                % normalem makes italics be italics, not underlines
    

    
    
    % Colors for the hyperref package
    \definecolor{urlcolor}{rgb}{0,.145,.698}
    \definecolor{linkcolor}{rgb}{.71,0.21,0.01}
    \definecolor{citecolor}{rgb}{.12,.54,.11}

    % ANSI colors
    \definecolor{ansi-black}{HTML}{3E424D}
    \definecolor{ansi-black-intense}{HTML}{282C36}
    \definecolor{ansi-red}{HTML}{E75C58}
    \definecolor{ansi-red-intense}{HTML}{B22B31}
    \definecolor{ansi-green}{HTML}{00A250}
    \definecolor{ansi-green-intense}{HTML}{007427}
    \definecolor{ansi-yellow}{HTML}{DDB62B}
    \definecolor{ansi-yellow-intense}{HTML}{B27D12}
    \definecolor{ansi-blue}{HTML}{208FFB}
    \definecolor{ansi-blue-intense}{HTML}{0065CA}
    \definecolor{ansi-magenta}{HTML}{D160C4}
    \definecolor{ansi-magenta-intense}{HTML}{A03196}
    \definecolor{ansi-cyan}{HTML}{60C6C8}
    \definecolor{ansi-cyan-intense}{HTML}{258F8F}
    \definecolor{ansi-white}{HTML}{C5C1B4}
    \definecolor{ansi-white-intense}{HTML}{A1A6B2}

    % commands and environments needed by pandoc snippets
    % extracted from the output of `pandoc -s`
    \providecommand{\tightlist}{%
      \setlength{\itemsep}{0pt}\setlength{\parskip}{0pt}}
    \DefineVerbatimEnvironment{Highlighting}{Verbatim}{commandchars=\\\{\}}
    % Add ',fontsize=\small' for more characters per line
    \newenvironment{Shaded}{}{}
    \newcommand{\KeywordTok}[1]{\textcolor[rgb]{0.00,0.44,0.13}{\textbf{{#1}}}}
    \newcommand{\DataTypeTok}[1]{\textcolor[rgb]{0.56,0.13,0.00}{{#1}}}
    \newcommand{\DecValTok}[1]{\textcolor[rgb]{0.25,0.63,0.44}{{#1}}}
    \newcommand{\BaseNTok}[1]{\textcolor[rgb]{0.25,0.63,0.44}{{#1}}}
    \newcommand{\FloatTok}[1]{\textcolor[rgb]{0.25,0.63,0.44}{{#1}}}
    \newcommand{\CharTok}[1]{\textcolor[rgb]{0.25,0.44,0.63}{{#1}}}
    \newcommand{\StringTok}[1]{\textcolor[rgb]{0.25,0.44,0.63}{{#1}}}
    \newcommand{\CommentTok}[1]{\textcolor[rgb]{0.38,0.63,0.69}{\textit{{#1}}}}
    \newcommand{\OtherTok}[1]{\textcolor[rgb]{0.00,0.44,0.13}{{#1}}}
    \newcommand{\AlertTok}[1]{\textcolor[rgb]{1.00,0.00,0.00}{\textbf{{#1}}}}
    \newcommand{\FunctionTok}[1]{\textcolor[rgb]{0.02,0.16,0.49}{{#1}}}
    \newcommand{\RegionMarkerTok}[1]{{#1}}
    \newcommand{\ErrorTok}[1]{\textcolor[rgb]{1.00,0.00,0.00}{\textbf{{#1}}}}
    \newcommand{\NormalTok}[1]{{#1}}
    
    % Additional commands for more recent versions of Pandoc
    \newcommand{\ConstantTok}[1]{\textcolor[rgb]{0.53,0.00,0.00}{{#1}}}
    \newcommand{\SpecialCharTok}[1]{\textcolor[rgb]{0.25,0.44,0.63}{{#1}}}
    \newcommand{\VerbatimStringTok}[1]{\textcolor[rgb]{0.25,0.44,0.63}{{#1}}}
    \newcommand{\SpecialStringTok}[1]{\textcolor[rgb]{0.73,0.40,0.53}{{#1}}}
    \newcommand{\ImportTok}[1]{{#1}}
    \newcommand{\DocumentationTok}[1]{\textcolor[rgb]{0.73,0.13,0.13}{\textit{{#1}}}}
    \newcommand{\AnnotationTok}[1]{\textcolor[rgb]{0.38,0.63,0.69}{\textbf{\textit{{#1}}}}}
    \newcommand{\CommentVarTok}[1]{\textcolor[rgb]{0.38,0.63,0.69}{\textbf{\textit{{#1}}}}}
    \newcommand{\VariableTok}[1]{\textcolor[rgb]{0.10,0.09,0.49}{{#1}}}
    \newcommand{\ControlFlowTok}[1]{\textcolor[rgb]{0.00,0.44,0.13}{\textbf{{#1}}}}
    \newcommand{\OperatorTok}[1]{\textcolor[rgb]{0.40,0.40,0.40}{{#1}}}
    \newcommand{\BuiltInTok}[1]{{#1}}
    \newcommand{\ExtensionTok}[1]{{#1}}
    \newcommand{\PreprocessorTok}[1]{\textcolor[rgb]{0.74,0.48,0.00}{{#1}}}
    \newcommand{\AttributeTok}[1]{\textcolor[rgb]{0.49,0.56,0.16}{{#1}}}
    \newcommand{\InformationTok}[1]{\textcolor[rgb]{0.38,0.63,0.69}{\textbf{\textit{{#1}}}}}
    \newcommand{\WarningTok}[1]{\textcolor[rgb]{0.38,0.63,0.69}{\textbf{\textit{{#1}}}}}
    
    
    % Define a nice break command that doesn't care if a line doesn't already
    % exist.
    \def\br{\hspace*{\fill} \\* }
    % Math Jax compatability definitions
    \def\gt{>}
    \def\lt{<}
    % Document parameters
    \title{12-Handout-DSO570}
    
    
    

    % Pygments definitions
    
\makeatletter
\def\PY@reset{\let\PY@it=\relax \let\PY@bf=\relax%
    \let\PY@ul=\relax \let\PY@tc=\relax%
    \let\PY@bc=\relax \let\PY@ff=\relax}
\def\PY@tok#1{\csname PY@tok@#1\endcsname}
\def\PY@toks#1+{\ifx\relax#1\empty\else%
    \PY@tok{#1}\expandafter\PY@toks\fi}
\def\PY@do#1{\PY@bc{\PY@tc{\PY@ul{%
    \PY@it{\PY@bf{\PY@ff{#1}}}}}}}
\def\PY#1#2{\PY@reset\PY@toks#1+\relax+\PY@do{#2}}

\expandafter\def\csname PY@tok@w\endcsname{\def\PY@tc##1{\textcolor[rgb]{0.73,0.73,0.73}{##1}}}
\expandafter\def\csname PY@tok@c\endcsname{\let\PY@it=\textit\def\PY@tc##1{\textcolor[rgb]{0.25,0.50,0.50}{##1}}}
\expandafter\def\csname PY@tok@cp\endcsname{\def\PY@tc##1{\textcolor[rgb]{0.74,0.48,0.00}{##1}}}
\expandafter\def\csname PY@tok@k\endcsname{\let\PY@bf=\textbf\def\PY@tc##1{\textcolor[rgb]{0.00,0.50,0.00}{##1}}}
\expandafter\def\csname PY@tok@kp\endcsname{\def\PY@tc##1{\textcolor[rgb]{0.00,0.50,0.00}{##1}}}
\expandafter\def\csname PY@tok@kt\endcsname{\def\PY@tc##1{\textcolor[rgb]{0.69,0.00,0.25}{##1}}}
\expandafter\def\csname PY@tok@o\endcsname{\def\PY@tc##1{\textcolor[rgb]{0.40,0.40,0.40}{##1}}}
\expandafter\def\csname PY@tok@ow\endcsname{\let\PY@bf=\textbf\def\PY@tc##1{\textcolor[rgb]{0.67,0.13,1.00}{##1}}}
\expandafter\def\csname PY@tok@nb\endcsname{\def\PY@tc##1{\textcolor[rgb]{0.00,0.50,0.00}{##1}}}
\expandafter\def\csname PY@tok@nf\endcsname{\def\PY@tc##1{\textcolor[rgb]{0.00,0.00,1.00}{##1}}}
\expandafter\def\csname PY@tok@nc\endcsname{\let\PY@bf=\textbf\def\PY@tc##1{\textcolor[rgb]{0.00,0.00,1.00}{##1}}}
\expandafter\def\csname PY@tok@nn\endcsname{\let\PY@bf=\textbf\def\PY@tc##1{\textcolor[rgb]{0.00,0.00,1.00}{##1}}}
\expandafter\def\csname PY@tok@ne\endcsname{\let\PY@bf=\textbf\def\PY@tc##1{\textcolor[rgb]{0.82,0.25,0.23}{##1}}}
\expandafter\def\csname PY@tok@nv\endcsname{\def\PY@tc##1{\textcolor[rgb]{0.10,0.09,0.49}{##1}}}
\expandafter\def\csname PY@tok@no\endcsname{\def\PY@tc##1{\textcolor[rgb]{0.53,0.00,0.00}{##1}}}
\expandafter\def\csname PY@tok@nl\endcsname{\def\PY@tc##1{\textcolor[rgb]{0.63,0.63,0.00}{##1}}}
\expandafter\def\csname PY@tok@ni\endcsname{\let\PY@bf=\textbf\def\PY@tc##1{\textcolor[rgb]{0.60,0.60,0.60}{##1}}}
\expandafter\def\csname PY@tok@na\endcsname{\def\PY@tc##1{\textcolor[rgb]{0.49,0.56,0.16}{##1}}}
\expandafter\def\csname PY@tok@nt\endcsname{\let\PY@bf=\textbf\def\PY@tc##1{\textcolor[rgb]{0.00,0.50,0.00}{##1}}}
\expandafter\def\csname PY@tok@nd\endcsname{\def\PY@tc##1{\textcolor[rgb]{0.67,0.13,1.00}{##1}}}
\expandafter\def\csname PY@tok@s\endcsname{\def\PY@tc##1{\textcolor[rgb]{0.73,0.13,0.13}{##1}}}
\expandafter\def\csname PY@tok@sd\endcsname{\let\PY@it=\textit\def\PY@tc##1{\textcolor[rgb]{0.73,0.13,0.13}{##1}}}
\expandafter\def\csname PY@tok@si\endcsname{\let\PY@bf=\textbf\def\PY@tc##1{\textcolor[rgb]{0.73,0.40,0.53}{##1}}}
\expandafter\def\csname PY@tok@se\endcsname{\let\PY@bf=\textbf\def\PY@tc##1{\textcolor[rgb]{0.73,0.40,0.13}{##1}}}
\expandafter\def\csname PY@tok@sr\endcsname{\def\PY@tc##1{\textcolor[rgb]{0.73,0.40,0.53}{##1}}}
\expandafter\def\csname PY@tok@ss\endcsname{\def\PY@tc##1{\textcolor[rgb]{0.10,0.09,0.49}{##1}}}
\expandafter\def\csname PY@tok@sx\endcsname{\def\PY@tc##1{\textcolor[rgb]{0.00,0.50,0.00}{##1}}}
\expandafter\def\csname PY@tok@m\endcsname{\def\PY@tc##1{\textcolor[rgb]{0.40,0.40,0.40}{##1}}}
\expandafter\def\csname PY@tok@gh\endcsname{\let\PY@bf=\textbf\def\PY@tc##1{\textcolor[rgb]{0.00,0.00,0.50}{##1}}}
\expandafter\def\csname PY@tok@gu\endcsname{\let\PY@bf=\textbf\def\PY@tc##1{\textcolor[rgb]{0.50,0.00,0.50}{##1}}}
\expandafter\def\csname PY@tok@gd\endcsname{\def\PY@tc##1{\textcolor[rgb]{0.63,0.00,0.00}{##1}}}
\expandafter\def\csname PY@tok@gi\endcsname{\def\PY@tc##1{\textcolor[rgb]{0.00,0.63,0.00}{##1}}}
\expandafter\def\csname PY@tok@gr\endcsname{\def\PY@tc##1{\textcolor[rgb]{1.00,0.00,0.00}{##1}}}
\expandafter\def\csname PY@tok@ge\endcsname{\let\PY@it=\textit}
\expandafter\def\csname PY@tok@gs\endcsname{\let\PY@bf=\textbf}
\expandafter\def\csname PY@tok@gp\endcsname{\let\PY@bf=\textbf\def\PY@tc##1{\textcolor[rgb]{0.00,0.00,0.50}{##1}}}
\expandafter\def\csname PY@tok@go\endcsname{\def\PY@tc##1{\textcolor[rgb]{0.53,0.53,0.53}{##1}}}
\expandafter\def\csname PY@tok@gt\endcsname{\def\PY@tc##1{\textcolor[rgb]{0.00,0.27,0.87}{##1}}}
\expandafter\def\csname PY@tok@err\endcsname{\def\PY@bc##1{\setlength{\fboxsep}{0pt}\fcolorbox[rgb]{1.00,0.00,0.00}{1,1,1}{\strut ##1}}}
\expandafter\def\csname PY@tok@kc\endcsname{\let\PY@bf=\textbf\def\PY@tc##1{\textcolor[rgb]{0.00,0.50,0.00}{##1}}}
\expandafter\def\csname PY@tok@kd\endcsname{\let\PY@bf=\textbf\def\PY@tc##1{\textcolor[rgb]{0.00,0.50,0.00}{##1}}}
\expandafter\def\csname PY@tok@kn\endcsname{\let\PY@bf=\textbf\def\PY@tc##1{\textcolor[rgb]{0.00,0.50,0.00}{##1}}}
\expandafter\def\csname PY@tok@kr\endcsname{\let\PY@bf=\textbf\def\PY@tc##1{\textcolor[rgb]{0.00,0.50,0.00}{##1}}}
\expandafter\def\csname PY@tok@bp\endcsname{\def\PY@tc##1{\textcolor[rgb]{0.00,0.50,0.00}{##1}}}
\expandafter\def\csname PY@tok@fm\endcsname{\def\PY@tc##1{\textcolor[rgb]{0.00,0.00,1.00}{##1}}}
\expandafter\def\csname PY@tok@vc\endcsname{\def\PY@tc##1{\textcolor[rgb]{0.10,0.09,0.49}{##1}}}
\expandafter\def\csname PY@tok@vg\endcsname{\def\PY@tc##1{\textcolor[rgb]{0.10,0.09,0.49}{##1}}}
\expandafter\def\csname PY@tok@vi\endcsname{\def\PY@tc##1{\textcolor[rgb]{0.10,0.09,0.49}{##1}}}
\expandafter\def\csname PY@tok@vm\endcsname{\def\PY@tc##1{\textcolor[rgb]{0.10,0.09,0.49}{##1}}}
\expandafter\def\csname PY@tok@sa\endcsname{\def\PY@tc##1{\textcolor[rgb]{0.73,0.13,0.13}{##1}}}
\expandafter\def\csname PY@tok@sb\endcsname{\def\PY@tc##1{\textcolor[rgb]{0.73,0.13,0.13}{##1}}}
\expandafter\def\csname PY@tok@sc\endcsname{\def\PY@tc##1{\textcolor[rgb]{0.73,0.13,0.13}{##1}}}
\expandafter\def\csname PY@tok@dl\endcsname{\def\PY@tc##1{\textcolor[rgb]{0.73,0.13,0.13}{##1}}}
\expandafter\def\csname PY@tok@s2\endcsname{\def\PY@tc##1{\textcolor[rgb]{0.73,0.13,0.13}{##1}}}
\expandafter\def\csname PY@tok@sh\endcsname{\def\PY@tc##1{\textcolor[rgb]{0.73,0.13,0.13}{##1}}}
\expandafter\def\csname PY@tok@s1\endcsname{\def\PY@tc##1{\textcolor[rgb]{0.73,0.13,0.13}{##1}}}
\expandafter\def\csname PY@tok@mb\endcsname{\def\PY@tc##1{\textcolor[rgb]{0.40,0.40,0.40}{##1}}}
\expandafter\def\csname PY@tok@mf\endcsname{\def\PY@tc##1{\textcolor[rgb]{0.40,0.40,0.40}{##1}}}
\expandafter\def\csname PY@tok@mh\endcsname{\def\PY@tc##1{\textcolor[rgb]{0.40,0.40,0.40}{##1}}}
\expandafter\def\csname PY@tok@mi\endcsname{\def\PY@tc##1{\textcolor[rgb]{0.40,0.40,0.40}{##1}}}
\expandafter\def\csname PY@tok@il\endcsname{\def\PY@tc##1{\textcolor[rgb]{0.40,0.40,0.40}{##1}}}
\expandafter\def\csname PY@tok@mo\endcsname{\def\PY@tc##1{\textcolor[rgb]{0.40,0.40,0.40}{##1}}}
\expandafter\def\csname PY@tok@ch\endcsname{\let\PY@it=\textit\def\PY@tc##1{\textcolor[rgb]{0.25,0.50,0.50}{##1}}}
\expandafter\def\csname PY@tok@cm\endcsname{\let\PY@it=\textit\def\PY@tc##1{\textcolor[rgb]{0.25,0.50,0.50}{##1}}}
\expandafter\def\csname PY@tok@cpf\endcsname{\let\PY@it=\textit\def\PY@tc##1{\textcolor[rgb]{0.25,0.50,0.50}{##1}}}
\expandafter\def\csname PY@tok@c1\endcsname{\let\PY@it=\textit\def\PY@tc##1{\textcolor[rgb]{0.25,0.50,0.50}{##1}}}
\expandafter\def\csname PY@tok@cs\endcsname{\let\PY@it=\textit\def\PY@tc##1{\textcolor[rgb]{0.25,0.50,0.50}{##1}}}

\def\PYZbs{\char`\\}
\def\PYZus{\char`\_}
\def\PYZob{\char`\{}
\def\PYZcb{\char`\}}
\def\PYZca{\char`\^}
\def\PYZam{\char`\&}
\def\PYZlt{\char`\<}
\def\PYZgt{\char`\>}
\def\PYZsh{\char`\#}
\def\PYZpc{\char`\%}
\def\PYZdl{\char`\$}
\def\PYZhy{\char`\-}
\def\PYZsq{\char`\'}
\def\PYZdq{\char`\"}
\def\PYZti{\char`\~}
% for compatibility with earlier versions
\def\PYZat{@}
\def\PYZlb{[}
\def\PYZrb{]}
\makeatother


    % Exact colors from NB
    \definecolor{incolor}{rgb}{0.0, 0.0, 0.5}
    \definecolor{outcolor}{rgb}{0.545, 0.0, 0.0}

    % Don't number sections
    \renewcommand{\thesection}{\hspace*{-0.5em}}
    \renewcommand{\thesubsection}{\hspace*{-0.5em}}



    
    % Prevent overflowing lines due to hard-to-break entities
    \sloppy 
    % Setup hyperref package
    \hypersetup{
      breaklinks=true,  % so long urls are correctly broken across lines
      colorlinks=true,
      urlcolor=urlcolor,
      linkcolor=linkcolor,
      citecolor=citecolor,
      }
    % Slightly bigger margins than the latex defaults
    
    \geometry{verbose,tmargin=1in,bmargin=1in,lmargin=1in,rmargin=1in}
    
    

    \begin{document}
    
    
    
    
    

    
    \hypertarget{session-12-simulation-modeling-i}{%
\section{Session 12: Simulation Modeling
I}\label{session-12-simulation-modeling-i}}

\hypertarget{generating-samples}{%
\subsection{1. Generating Samples}\label{generating-samples}}

	
\begin{Verbatim}[commandchars=\\\{\}]
{\color{incolor}[{\color{incolor}1}]:} \PY{k+kn}{from} \PY{n+nn}{scipy}\PY{n+nn}{.}\PY{n+nn}{stats} \PY{k}{import} \PY{n}{norm}
     \PY{k+kn}{import} \PY{n+nn}{numpy} \PY{k}{as} \PY{n+nn}{np}
     \PY{n}{np}\PY{o}{.}\PY{n}{random}\PY{o}{.}\PY{n}{seed}\PY{p}{(}\PY{l+m+mi}{0}\PY{p}{)}
     \PY{n}{dist}\PY{o}{=}\PY{n}{norm}\PY{p}{(}\PY{l+m+mi}{100}\PY{p}{,}\PY{l+m+mi}{30}\PY{p}{)}
     \PY{n}{dist}\PY{o}{.}\PY{n}{rvs}\PY{p}{(}\PY{p}{)}
\end{Verbatim}
	

    
    
\begin{verbatim}
152.92157037902993
\end{verbatim}

    

	
\begin{Verbatim}[commandchars=\\\{\}]
{\color{incolor}[{\color{incolor}2}]:} \PY{n}{dist}\PY{o}{.}\PY{n}{rvs}\PY{p}{(}\PY{n}{size}\PY{o}{=}\PY{l+m+mi}{3}\PY{p}{)}
\end{Verbatim}
	

    
    
\begin{verbatim}
array([112.00471625, 129.36213952, 167.22679598])
\end{verbatim}

    

	
\begin{Verbatim}[commandchars=\\\{\}]
{\color{incolor}[{\color{incolor}3}]:} \PY{k+kn}{import} \PY{n+nn}{pandas} \PY{k}{as} \PY{n+nn}{pd}
     \PY{n}{data}\PY{o}{=}\PY{n}{dist}\PY{o}{.}\PY{n}{rvs}\PY{p}{(}\PY{n}{size}\PY{o}{=}\PY{l+m+mi}{10000}\PY{p}{)}
     \PY{n}{demand}\PY{o}{=}\PY{n}{pd}\PY{o}{.}\PY{n}{Series}\PY{p}{(}\PY{n}{data}\PY{p}{)}
     \PY{n}{demand}\PY{o}{.}\PY{n}{head}\PY{p}{(}\PY{p}{)}
\end{Verbatim}
	

    
    
\begin{verbatim}
0    156.026740
1     70.681664
2    128.502653
3     95.459284
4     96.903434
dtype: float64
\end{verbatim}

    

	
\begin{Verbatim}[commandchars=\\\{\}]
{\color{incolor}[{\color{incolor}4}]:} \PY{n}{demand}\PY{o}{.}\PY{n}{mean}\PY{p}{(}\PY{p}{)}
\end{Verbatim}
	

    
    
\begin{verbatim}
99.43350357671024
\end{verbatim}

    

	
\begin{Verbatim}[commandchars=\\\{\}]
{\color{incolor}[{\color{incolor}5}]:} \PY{n}{np}\PY{o}{.}\PY{n}{mean}\PY{p}{(}\PY{n}{data}\PY{p}{)}
\end{Verbatim}
	

    
    
\begin{verbatim}
99.43350357671024
\end{verbatim}

    

	
\begin{Verbatim}[commandchars=\\\{\}]
{\color{incolor}[{\color{incolor}6}]:} \PY{n}{demand}\PY{o}{.}\PY{n}{std}\PY{p}{(}\PY{p}{)}
\end{Verbatim}
	

    
    
\begin{verbatim}
29.619606530616284
\end{verbatim}

    

	
\begin{Verbatim}[commandchars=\\\{\}]
{\color{incolor}[{\color{incolor}7}]:} \PY{n}{np}\PY{o}{.}\PY{n}{std}\PY{p}{(}\PY{n}{data}\PY{p}{)}
\end{Verbatim}
	

    
    
\begin{verbatim}
29.618125513263394
\end{verbatim}

    

	
\begin{Verbatim}[commandchars=\\\{\}]
{\color{incolor}[{\color{incolor}8}]:} \PY{p}{(}\PY{n}{demand}\PY{o}{\PYZlt{}}\PY{l+m+mi}{100}\PY{p}{)}\PY{o}{.}\PY{n}{mean}\PY{p}{(}\PY{p}{)}
\end{Verbatim}
	

    
    
\begin{verbatim}
0.5109
\end{verbatim}

    

	
\begin{Verbatim}[commandchars=\\\{\}]
{\color{incolor}[{\color{incolor}9}]:} \PY{n}{np}\PY{o}{.}\PY{n}{mean}\PY{p}{(}\PY{n}{data}\PY{o}{\PYZlt{}}\PY{l+m+mi}{100}\PY{p}{)}
\end{Verbatim}
	

    
    
\begin{verbatim}
0.5109
\end{verbatim}

    

	
\begin{Verbatim}[commandchars=\\\{\}]
{\color{incolor}[{\color{incolor}33}]:} \PY{n}{demand}\PY{o}{.}\PY{n}{hist}\PY{p}{(}\PY{n}{bins}\PY{o}{=}\PY{l+m+mi}{20}\PY{p}{)}
\end{Verbatim}
	

    

    \begin{center}
    \adjustimage{max size={0.4\linewidth}{0.9\paperheight}}{12-Handout-DSO570_files/12-Handout-DSO570_10_1.png}
    \end{center}

    
	
\begin{Verbatim}[commandchars=\\\{\}]
{\color{incolor}[{\color{incolor}11}]:} \PY{n}{demand}\PY{o}{.}\PY{n}{hist}\PY{p}{(}\PY{n}{bins}\PY{o}{=}\PY{l+m+mi}{20}\PY{p}{,}\PY{n}{density}\PY{o}{=}\PY{k+kc}{True}\PY{p}{,}\PY{n}{cumulative}\PY{o}{=}\PY{k+kc}{True}\PY{p}{)}
\end{Verbatim}
	

    
    
    

    \begin{center}
    \adjustimage{max size={0.4\linewidth}{0.9\paperheight}}{12-Handout-DSO570_files/12-Handout-DSO570_11_1.png}
    \end{center}

    
	
\begin{Verbatim}[commandchars=\\\{\}]
{\color{incolor}[{\color{incolor}12}]:} \PY{n}{distA}\PY{o}{=}\PY{n}{norm}\PY{p}{(}\PY{l+m+mi}{200}\PY{p}{,}\PY{l+m+mi}{50}\PY{p}{)}
      \PY{n}{distB}\PY{o}{=}\PY{n}{norm}\PY{p}{(}\PY{l+m+mi}{100}\PY{p}{,}\PY{l+m+mi}{30}\PY{p}{)}
      \PY{k+kn}{from} \PY{n+nn}{scipy}\PY{n+nn}{.}\PY{n+nn}{stats} \PY{k}{import} \PY{n}{bernoulli}
      \PY{n}{distSegment}\PY{o}{=}\PY{n}{bernoulli}\PY{p}{(}\PY{l+m+mf}{0.4}\PY{p}{)}
      \PY{n}{data}\PY{o}{=}\PY{p}{[}\PY{p}{]}
      \PY{k}{for} \PY{n}{i} \PY{o+ow}{in} \PY{n+nb}{range}\PY{p}{(}\PY{l+m+mi}{10000}\PY{p}{)}\PY{p}{:}
          \PY{n}{segment}\PY{o}{=}\PY{n}{distSegment}\PY{o}{.}\PY{n}{rvs}\PY{p}{(}\PY{p}{)}
          \PY{k}{if} \PY{n}{segment}\PY{o}{==}\PY{l+m+mi}{1}\PY{p}{:}
              \PY{n}{value}\PY{o}{=}\PY{n}{distA}\PY{o}{.}\PY{n}{rvs}\PY{p}{(}\PY{p}{)}
          \PY{k}{else}\PY{p}{:}
              \PY{n}{value}\PY{o}{=}\PY{n}{distB}\PY{o}{.}\PY{n}{rvs}\PY{p}{(}\PY{p}{)}
          \PY{n}{data}\PY{o}{.}\PY{n}{append}\PY{p}{(}\PY{n}{value}\PY{p}{)}
      \PY{n}{demand2}\PY{o}{=}\PY{n}{pd}\PY{o}{.}\PY{n}{Series}\PY{p}{(}\PY{n}{data}\PY{p}{)}
      \PY{n}{demand2}\PY{o}{.}\PY{n}{hist}\PY{p}{(}\PY{n}{bins}\PY{o}{=}\PY{l+m+mi}{20}\PY{p}{)}
\end{Verbatim}
	

        

    \begin{center}
    \adjustimage{max size={0.5\linewidth}{0.9\paperheight}}{12-Handout-DSO570_files/12-Handout-DSO570_12_1.png}
    \end{center}

    
	
\begin{Verbatim}[commandchars=\\\{\}]
{\color{incolor}[{\color{incolor}13}]:} \PY{n}{values}\PY{o}{=}\PY{p}{[}\PY{l+m+mi}{1}\PY{p}{,}\PY{l+m+mf}{2.5}\PY{p}{,}\PY{l+m+mf}{3.5}\PY{p}{]}
      \PY{n}{probs}\PY{o}{=}\PY{p}{[}\PY{l+m+mf}{0.3}\PY{p}{,}\PY{l+m+mf}{0.5}\PY{p}{,}\PY{l+m+mf}{0.2}\PY{p}{]}
      \PY{n}{np}\PY{o}{.}\PY{n}{random}\PY{o}{.}\PY{n}{choice}\PY{p}{(}\PY{n}{values}\PY{p}{,}\PY{n}{p}\PY{o}{=}\PY{n}{probs}\PY{p}{)}
\end{Verbatim}
	

    
    
\begin{verbatim}
1.0
\end{verbatim}

    

	
\begin{Verbatim}[commandchars=\\\{\}]
{\color{incolor}[{\color{incolor}14}]:} \PY{n}{np}\PY{o}{.}\PY{n}{random}\PY{o}{.}\PY{n}{choice}\PY{p}{(}\PY{n}{values}\PY{p}{,}\PY{n}{p}\PY{o}{=}\PY{n}{probs}\PY{p}{,}\PY{n}{size}\PY{o}{=}\PY{l+m+mi}{10}\PY{p}{)}
\end{Verbatim}
	

    
    
\begin{verbatim}
array([2.5, 3.5, 1. , 2.5, 1. , 2.5, 1. , 2.5, 1. , 1. ])
\end{verbatim}

    

	
\begin{Verbatim}[commandchars=\\\{\}]
{\color{incolor}[{\color{incolor}15}]:} \PY{n}{samples}\PY{o}{=}\PY{n}{pd}\PY{o}{.}\PY{n}{Series}\PY{p}{(}\PY{n}{np}\PY{o}{.}\PY{n}{random}\PY{o}{.}\PY{n}{choice}\PY{p}{(}\PY{n}{values}\PY{p}{,}\PY{n}{p}\PY{o}{=}\PY{n}{probs}\PY{p}{,}\PY{n}{size}\PY{o}{=}\PY{l+m+mi}{10000}\PY{p}{)}\PY{p}{)}
      \PY{n}{samples}\PY{o}{.}\PY{n}{hist}\PY{p}{(}\PY{p}{)}
\end{Verbatim}
	
   

    \begin{center}
    \adjustimage{max size={0.4\linewidth}{0.9\paperheight}}{12-Handout-DSO570_files/12-Handout-DSO570_15_1.png}
    \end{center}
    
	
\begin{Verbatim}[commandchars=\\\{\}]
{\color{incolor}[{\color{incolor}16}]:} \PY{n}{samples}\PY{o}{.}\PY{n}{hist}\PY{p}{(}\PY{n}{bins}\PY{o}{=}\PY{l+m+mi}{100}\PY{p}{,}\PY{n}{density}\PY{o}{=}\PY{k+kc}{True}\PY{p}{,}\PY{n}{cumulative}\PY{o}{=}\PY{k+kc}{True}\PY{p}{)}
\end{Verbatim}
	

        
   

    \begin{center}
    \adjustimage{max size={0.4\linewidth}{0.9\paperheight}}{12-Handout-DSO570_files/12-Handout-DSO570_16_1.png}
    \end{center}
    
    
    \textbf{Q1-a:} Generate 10000 samples of a uniform distribution between
1 and 3 and plot the histogram, as well as the empirical CDF. Calculate
the mean and standard deviation of the samples, as well as the
proportion between 2 and 2.5 (inclusive).

    \textbf{Q1-b:} Generate 100 samples of a binomial distrubution with
\(n=10\) and \(p=0.3\). Calculate the mean and standard deviation of the
sample and compare with what it should be from the formula. Plot a
histogram with 50 bins. Repeat with 10,000 samples.

    \textbf{Q2:} You are tasked with forecasting demand for a new product.
Based on past data and your knowledge of the product, you estimate that
the product quality will be Amazing with probability 0.1, Mediocre with
probability 0.5, and Terrible with probability 0.4. You model the demand
as normally distributed, with mean and standard deviation depending on
the product quality as follows.

\begin{longtable}[]{@{}llll@{}}
\toprule
Prod. Quality: & Amazing & Mediocre & Terrible\tabularnewline
\midrule
\endhead
\(\mu\) & 10000 & 5000 & 1000\tabularnewline
\(\sigma\) & 2000 & 1000 & 300\tabularnewline
\bottomrule
\end{longtable}

Create a \texttt{Series} called ``forecast'' with 10,000 samples of the
demand forecast, and compute the mean and standard deviation of the
samples, as well as the probability that demand is more than 6000.
Finally, plot a histogram of the samples with 100 bins, as well as the
empirical CDF.

	
\begin{Verbatim}[commandchars=\\\{\}]
{\color{incolor}[{\color{incolor}24}]:} 
\end{Verbatim}
\begin{Verbatim}[commandchars=\\\{\}]
Sample mean: 3881.302260377957
Sample standard deviation: 2939.8432697511016
Probability demand more than 6000: 0.1781

\end{Verbatim}

	

    
    

    \begin{center}
    \adjustimage{max size={0.5\linewidth}{0.9\paperheight}}{12-Handout-DSO570_files/12-Handout-DSO570_20_2.png}
    \end{center}
   
	
\begin{Verbatim}[commandchars=\\\{\}]
{\color{incolor}[{\color{incolor}25}]:} 
\end{Verbatim}
	

    
    

    \begin{center}
    \adjustimage{max size={0.5\linewidth}{0.9\paperheight}}{12-Handout-DSO570_files/12-Handout-DSO570_21_1.png}
    \end{center}

    
    \textbf{Q3:} Nadeem is a car salesperson who faces the following
incentive scheme at the dealership where he works. For each month, there
is a ``target profit'' that the dealership sets for the month. If he
makes more profit for the dealership that month than the target, then he
receives a 20\% bonus on the amount of profit over the target. However,
if he does not meet the target, he receives no bonus. For example, if
the target is 80,000 and he makes 100,000 of profit, then he receives a
4,000 bonus that month. However, if he makes 70,000, then he receives
zero bonus that month. Nadeem would like to understand the distribution
of his monthly bonus.

Nadeem estimates that the number of cars he sells is binomial
distributed with \(n=200\) and \(p=0.2\). On every car he sells, the
amount of profit he makes for the dealership is normally distributed
with \(\mu=3000\) and \(\sigma=1000\), and the profit from each car is
independent of another.

Create a \texttt{Series} called ``monthlyBonus'' with 10,000 samples of
his monthly bonus. Compute the mean, the standard deviation, the
probability the bonus is less than 5000, and plot a histogram with 50
bins as well as the empirical CDF.

	
\begin{Verbatim}[commandchars=\\\{\}]
{\color{incolor}[{\color{incolor}26}]:} 
\end{Verbatim}
\begin{Verbatim}[commandchars=\\\{\}]
Mean is 7966.691715092367
Standard deviation is 3595.714686711149
Probability less than 5000 is 0.2058

\end{Verbatim}

	

    
   

    \begin{center}
    \adjustimage{max size={0.6\linewidth}{0.9\paperheight}}{12-Handout-DSO570_files/12-Handout-DSO570_23_2.png}
    \end{center}
    
	
\begin{Verbatim}[commandchars=\\\{\}]
{\color{incolor}[{\color{incolor}27}]:} 
\end{Verbatim}
	

    
   

    \begin{center}
    \adjustimage{max size={0.5\linewidth}{0.9\paperheight}}{12-Handout-DSO570_files/12-Handout-DSO570_24_1.png}
    \end{center}

    
    \textbf{(Bonus)} This question asks you to illustrate the Central Limit
Theorem (CLT) by example. Consider the following distribution,
\[X = \begin{cases} 3 & \text{ with probability $0.4$,} \\ 5 & \text{ with probability $0.1$,} \\ 9 & \text{ with probability $0.5$.} \end{cases}\]

Define \(Y_n\) to be the sum of \(n\) independent random variables with
the above distribution. Create 10000 samples of \(Y_1\), \(Y_5\),
\(Y_{30}\), \(Y_{100}\), and \(Y_{1000}\) and plot their histograms
(with 30 bins). You should be able to see the histograms converging to a
Bell curve as \(n\) increases.

The Central Limit Theorem (CLT) says that this phenomenon always
happens, regardless of the distribution of \(X\). Moreover, it still
takes place even if the \(n\) independent random variables do not have
the same distribution, as long as each term in the sum is ``small''
relative to the whole. (For a precise mathematical formulation of the
CLT in the case with non-identical random variables, search for Lyapunov
or Lindeberg CLT on Wikipedia.)


    % Add a bibliography block to the postdoc
    
    
    
    \end{document}
